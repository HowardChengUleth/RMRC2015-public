\problemname{Space Junk}


\setlength{\columnsep}{15pt}
\illustration{0.25}{junk.png}

% https://openclipart.org/detail/16813/satellite
% public domain


According to NASA's web page, there are more than $500\,000$ pieces of
``space junk'' that are tracked.  Care must be taken in mission
planning so satellites and other spacecrafts do not collide with
these pieces of space junk.

For this problem, we will consider the simplified case in which both
the spacecraft and the space junk can be modelled as spheres that are
travelling in a straight line.  Given the current locations of the two
spheres as well as their velocities, when would they collide in the
future, if ever?

\section*{Input}
The first line of input contains a single positive integer $T \leq
500$ indicating the number of test cases.  Each test case is specified
by two lines.  The first line specifies the sphere representing the
spacecraft, while the second line specifies the sphere representing
the space junk.  Each sphere is specified by the seven integers $ x,
y, z, r, v_x, v_y, v_z$.  The center of the sphere is currently
located at $(x,y,z)$, the radius of the sphere is $r$, and the sphere
is travelling along the direction specified by the vector $(v_x,v_y,
v_z)$.  If the vector is $(0,0,0)$, the sphere is stationary.

The absolute value of all integers are at most 100, and $r$ is
positive.  All coordinates and radius are measured in meters, and the
velocities are measured in meters/second.  

You may assume that the two spheres do not touch each other initially.

\section*{Output}

For each test case, output a line containing the time (in seconds) at
which the spacecraft first collides with the space junk.  If they
never collide, print \verb|No collision| instead.  Answers within 0.01
of the correct result are acceptable.

